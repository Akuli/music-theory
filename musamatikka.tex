\documentclass[12pt]{article}

\usepackage[utf8]{inputenc}
\usepackage[T1]{fontenc}
\usepackage[finnish]{babel}

\usepackage{amsfonts}
\usepackage{amssymb}
\usepackage{amsmath}
\usepackage{amsthm}
\usepackage{enumerate}
\usepackage{graphicx}
\usepackage{float}
\usepackage{fullpage}
\usepackage{media9}
\usepackage{attachfile}

\theoremstyle{plain}
\newtheorem{theorem}{Lause}
\newtheorem{lemma}[theorem]{Lemma}
%\newtheorem{corollary}{Seuraus}

\theoremstyle{definition}
\newtheorem{definition}[theorem]{M\"a\"aritelm\"a}
\newtheorem{example}[theorem]{Esimerkki}

\newcommand{\tehtava}[1]{$$ $$\textbf{Tehtävä {#1}.}}
\newcommand{\abs}[1]{\left|{#1}\right|}
\newcommand{\normi}[1]{\left\|{#1}\right\|}
\newcommand{\epsi}{\varepsilon}
\newcommand{\st}[2]{\left({#1}\middle|{#2}\right)}
\newcommand{\con}[1]{\overline{#1}}
\newcommand{\cl}[1]{\overline{#1}}
\newcommand{\re}{\operatorname{Re}}
\newcommand{\im}{\operatorname{Im}}
\newcommand{\T}{\operatorname{T}}
\newcommand{\del}{\partial}
\newcommand{\id}{\operatorname{id}}
\newcommand{\pr}{\operatorname{pr}}
\newcommand{\A}{\operatorname{A}}
\newcommand{\B}{\operatorname{B}}
\newcommand{\C}{\operatorname{C}}
\newcommand{\D}{\operatorname{D}}
\newcommand{\E}{\operatorname{E}}
\newcommand{\F}{\operatorname{F}}
\newcommand{\G}{\operatorname{G}}

\newcounter{audiocounter}
\newcommand{\audio}[1]{
    \stepcounter{audiocounter}
    % TODO: try caching the wavs in latex?
    % I tried caching within the python script and it didn't speed up.
    \input{| mkdir -p wavs && python3 make_audio.py '#1' -o wavs/\theaudiocounter.wav}
}
\newcommand{\Audio}[1]{\begin{center}\audio{#1}\end{center}}

\begin{document}

\section*{Intervallit}

\begin{definition}
\emph{Intervalli} tarkoittaa kahden eri äänenkorkeuden eroa.
Tärkeimmät intervallit ovat
\begin{itemize}
\item \emph{oktaavi}: \audio{C4 C5}
\item \emph{sävelaskel} eli \emph{kokosävelaskel} eli \emph{kokonainen sävelaskel}
on seuraavien äänenten välinen intervalli: \audio{C4 D4}
\item \emph{puolisävelaskel} eli \emph{puolikas sävelaskel} (puolet kokonaisesta sävelaskeleesta): \audio{C4 C#4}
\end{itemize}
\end{definition}

Äänien kuuntelu klikkaamalla \textbf{ei toimi selaimessa}.
Joudut luultavasti lataamaan äänitiedoston omalle koneelle.

Oktaavi on jossakin mielessä sama ääni eri korkeudella.
Tähän palataan myöhemmin.

Huomaa, että sama intervalli löytyy usiden eri äänien välitä.
Esimerkiksi nämä ovat oktaaveja:
\Audio{E5 E6} \Audio{Ab3 Ab4}
Nämä taas ovat kokonaisia sävelaskelia:
\Audio{E5 F#5} \Audio{Ab3 Bb3}
Seuraava lause on tärkeä, koska “taikaluku” 12 esiintyy jatkossa paljon.

\begin{theorem}
Yksi oktaavi on 12 puolikasta sävelaskelta.
\end{theorem}
\begin{proof}
Kuunnellaan tämä ääni: \Audio{C4 D4 E4 F#4 G#4 A#4 C5 . C4 C5}
Tässä on ensin seitsemän nousevan äänen jono.
Näiden väleissä on 6 nousua, ja jokainen nousu on yhden sävelaskelen kokoinen.
Lopuksi ensimmäinen ja viimeinen ääni soivat uudestaan, ja huomataan, että ne muodostavat oktaavin.
Näin ollen oktaavi on 6 kokonaista sävelaskelta eli 12 puolikasta.
\end{proof}

TODO: selitä miksi oktaavi on jaettu juuri 12 osaan?

Saman todistuksen voi tehdä puolikkailla sävelaskelilla,
mutta tällöin on vaikeampi vakuuttua siitä,
että jokainen nousu on tarkalleen puolisävelaskelen kokoinen:
\Audio{C4 C#4 D4 D#4 E4 F4 F#4 G4 G#4 A4 A#4 B4 C5 . C4 C5}

Puolikkaista sävelaskeleista voidaan muodostaa muitakin intervalleja:
\newpage
\begin{itemize} \itemsep 0em
\item 1 puolikas: \audio{C4 C#4}
\item 2 puolikasta (1 kokonainen): \audio{C4 D4}
\item 3 puolikasta: \audio{C4 D#4}
\item 4 puolikasta: \audio{C4 E4}
\item 5 puolikasta: \audio{C4 F4}
\item 6 puolikasta: \audio{C4 F#4}
\item 7 puolikasta: \audio{C4 G4}
\item 8 puolikasta: \audio{C4 G#4}
\item 9 puolikasta: \audio{C4 A4}
\item 10 puolikasta: \audio{C4 A#4}
\item 11 puolikasta: \audio{C4 B4}
\item 12 puolikasta (oktaavi): \audio{C4 C5}
\end{itemize}
Näille on annettu nimet, mutta nimistä ei kannata välittää.
Intervallien kuuleminen on paljon tärkeämpää, ja \textbf{sitä kannattaa harjoitella}.
Voit vaikka klikata listasta silmät kiinni randomia ääntä
ja yrittää arvata, kuinka monen puolikkaan intervalli on kyseessä.
Tämän pitäisi onnistua helposti.


\section*{Äänenkorkeudet kokonaislukuina}

Länsimaisessa musiikissa kaikki äänet ovat lähes aina jonkin puolisävelaskelen monikerran päässä toisistaan.
Tämä johtuu soittimista.
Esimerkiksi kitarassa sormen liikuttaminen seuraavalle nauhalle muuttaa kielen ääntä puolisävelaskelen verran,
ja pianon koskettimet ovat puolisävelaskelen välein, jos lasketaan mukaan mustat ja valkoiset koskettimet.
Suuri osa musiikin teoriasta palautuu kokonaisluvuilla laskemiseksi, kun rajoitumme näihin äänenkorkeuksiin.
\begin{definition}
Merkitään intervalleja luonnollisina lukuina käyttäen yksikkönä puolisävelaskelta,
eli luku $n \in \mathbb{N}$ vastaa $n$ puolikkaan sävelaskelen kokoista intervallia.
Olkoon $a$ äänenkorkeus ja $n \in \mathbb{Z}$.
Merkintä $a+n$ tarkoittaa sitä äänenkorkeutta, joka on $n$ puolikasta sävelaskelta ylempi kuin $b$.
Negatiivinen $n$ tarkoittaa, että $a+n$ on matalampi ääni kuin $b$.
\end{definition}

Jos esimerkiksi $a$ on \audio{A4}, niin $a+3$ on \audio{C5}.
Tässä siis ajatellaan intervalleja kokonaislukuna: äänten $a$ ja $b$ välinen intervalli on $3$ (puolisävelaskelta).
\begin{example}
Tässä on pätkä Ukko Nooaa. Selvitetään sen äänet ensimmäiseen ääneen verrattuna.
\Audio{C4 C4 C4 E4 D4 D4 D4 F4 E4 E4 D4 D4 C4}
Kuunnellaan ensin neljää ensimmäistä ääntä: \Audio{C4 C4 C4 E4}
Olkoon $s$ ensimmäinen ääni. Se tulee kolme kertaa.
Seuraava ääni on neljän puolikkaan päässä ensimmäisestä (vertaa yllä olevaan intervallilistaan),
joten se on $s+4$.
$$
s,~ s,~ s,~ s+4
$$
Otetaan mukaan seuraava ääni, joka tulee kolmeen kertaan. \Audio{C4 C4 C4 E4 D4 D4 D4}
Se on edellisten äänien puolessa välissä, siis $s+2$:
$$
s,~ s,~ s,~ s+4,~ s+2,~ s+2,~ s+2
$$
Näiden jälkeen tuleva ääni ei ole $s+6$ kuten voisi arvata,
sillä sen kanssa Ukko Nooan alku kuulostaisi väärältä:
\Audio{C4 C4 C4 E4 D4 D4 D4 F#4}
Oikea ääni on $s+5$, ja tämän jälkeen toistetaankin samoja ääniä kuin on aikaisemmin ollut.
Ukko nooan äänet ovat siis:
$$
s,~ s,~ s,~ s+4,~ s+2,~ s+2,~ s+2,~ s+5,~ s+4,~ s+4,~ s+2,~ s+2,~ s.
$$
\end{example}
Lukua $s$ sanotaan \emph{sävellajiksi}.
Sävellajin määritteleminen täsmällisesti on vaikeaa.
Sitä kannattaa ajatella äänenä, johon muiden äänien vertaaminen on luontevaa,
ja se on yleensä ensimmäinen tai viimeinen ääni.
Eri sävellajilla $s$ koko Ukko Nooa siirtyy saman verran ylöspäin tai alaspäin,
vaikkapa näin:
\Audio{F#4 F#4 F#4 A#4 G#4 G#4 G#4 B4 A#4 A#4 G#4 G#4 F#4}

Äänenkorkeuksiakin voi ajatella kokonaislukuina,
jos sovitaan, mitä tarkoittaa vaikkapa äänenkorkeus 0
ja ilmaistaan muut äänenkorkeudet vertaamalla siihen.
Näin päästään puhtaasti kokonaisluvuilla laskemiseen ja merkinnät lyhenevät.
\begin{example}
Jos sovitaan, että Ukko Nooan ensimmäinen ääni on nolla, niin kaikki äänet ovat
$$
0,0,0,4,2,2,2,5,4,4,2,2,0.
$$
\end{example}
(Tämä näyttää nyt epäilyttävästi tiktokeilta,
joissa näytetään selvästi feikkiä soittoa ja kasa randomeja numeroita.
Ehkä tiktokeissa on alun perin yritetty tehdä jotain tämän kaltaista.)

\begin{definition}
Nimetään äänenkorkeudet näin:
\begin{itemize}
\item 440 Hz värähtelystä muodostuvan äänen korkeuden nimi on $A4$: \audio{A4}
\item $B_4 = A_4 + 2$
\item $C_5 = B_4 + 1$
\item $D_5 = C_4 + 2$
\item $E_5 = D_4 + 2$
\item $F_5 = E_4 + 1$
\item $G_5 = F_4 + 1$
\item Numeron kasvattaminen yhdellä muuttaa äänenkorkeutta oktaavin verran ylöspäin.
\item $\#$ nimessä tarkoittaa puolisävelaskelen verran korkeampaa ääntä.
\item $b$ nimessä tarkoittaa puolisävelaskelen verran matalampaa ääntä.
\end{itemize}
\end{definition}
Kaikki äänenkorkeudet puolisävelaskelen välein ovat siis
\begin{align*}
&\vdots \\
C_2,~C\#_2,~D_2,~D\#_2,~E_2,~F_2,~F\#_2,~G_2,~G\#_2,~A_2,~A\#_2,~B_2,~\\
C_3,~C\#_3,~D_3,~D\#_3,~E_3,~F_3,~F\#_3,~G_3,~G\#_3,~A_3,~A\#_3,~B_3,~\\
C_4,~C\#_4,~D_4,~D\#_4,~E_4,~F_4,~F\#_4,~G_4,~G\#_4,~A_4,~A\#_4,~B_4,~\\
&\vdots
\end{align*}
tai ekvivalentisti
\begin{align*}
&\vdots \\
\C2,~\D b2,~\D2,~\D2\#,~\E2,~\F2,~\F2\#,~\G2,~\G2\#,~\A2,~\A2\#,~\B2,~\\
\C3,~\C3\#,~\D3,~\D3\#,~\E3,~\F3,~\F3\#,~\G3,~\G3\#,~\A3,~\A3\#,~\B3,~\\
\C4,~\C4\#,~\D4,~\D4\#,~\E4,~\F4,~\F4\#,~\G4,~\G4\#,~\A4,~\A4\#,~\B4,~\\
&\vdots
\end{align*}


\end{document}
